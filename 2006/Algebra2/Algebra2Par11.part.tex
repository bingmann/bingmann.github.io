\chapter{Multilineare Algebra}

\section{Moduln}

Sei $R$ ein (kommutativer) Ring (mit Eins) (in der ganzen Vorlesung).

\begin{Def}
\label{1.1}
  \begin{enumerate}
    \item Eine abelsche Gruppe $(M,+)$ zusammen mit einer Abbildung
          $\cdot : R \times M \to M$ heißt \emp{$R$-Modul}\index{R-Modul} (genauer:
          $R$-Linksmodul), wenn gilt:
          \begin{enumerate}
            \item[(i)] $a \cdot (x+y) = a \cdot x + a \cdot y$
            \item[(ii)] $(a+b) \cdot x = a \cdot x + b \cdot x$
            \item[(iii)] $(a \cdot b) \cdot x = a \cdot (b \cdot x)$
            \item[(iv)] $1 \cdot x = x$
          \end{enumerate}
          für alle $a,b \in R,\;x,y \in M$.
    \item Eine Abbildung $\varphi: M \to M'$ zwischen $R$-Modulen $M$, $M'$
          heißt \emp{$R$-Modul-Homo\-morphismus}\index{R-Modul!-Homomorphismus} (kurz
          \emp{$R$-linear}\index{R-linear}), wenn für alle $x,y \in M, \; a,b \in R$
          gilt:\\
          $\varphi (a \cdot x + b \cdot y) = a \cdot \varphi (x) + b \cdot
          \varphi (y)$
  \end{enumerate}
\end{Def}

\begin{nnBsp}
  \begin{enumerate}
    \item[(1)] $R = K$ Körper. Dann ist $R$-Modul = $K$-Vektorraum und
               $R$-linear = linear
    \item[(2)] $R$ ist $R$-Modul. Jedes Ideal $I \subseteq R$ ist $R$-Modul
    \item[(3)] Jede abelsche Gruppe ist ein $\mathbb{Z}$-Modul.\\
               (denn: $n \cdot x = \underbrace{x + x + \cdots + x}_{n\text{-mal}}$
               definiert die Abbildung $\cdot: \mathbb{Z} \times M \to M$
               wie in \ref{1.1} gefordert)
  \end{enumerate}
\end{nnBsp}

\begin{BemDef}
  \begin{enumerate}
    \item Sind $M,M'$ $R$-Moduln, so ist Hom$_R(M,M') = \{\varphi: M \to M' :
          \varphi \text{ ist } R\text{-linear}\}$ ein $R$-Modul durch
          $(\varphi_1 + \varphi_2)(x) = \varphi_1(x) + \varphi_2(x)$ und
          $(a \cdot \varphi_1)(x) = a \cdot \varphi_1(x)$.
    \item $M^* = \Hom_R(M,R)$ heißt dualer Modul.\index{R-Modul!dualer}
  \end{enumerate}
\end{BemDef}

% fuer das grosse kartesische produkt im naechsten absatz
\newcommand{\BIGOP}[1]{\mathop{\mathchoice%
{\raise-0.22em\hbox{\huge $#1$}}%
{\raise-0.05em\hbox{\Large $#1$}}{\hbox{\large $#1$}}{#1}}}
\newcommand{\bigtimes}{\BIGOP{\times}} 

\begin{nnBsp}
  $R = \mathbb{Z}$\\
  Hom$_R(\mathbb{Z}/2\mathbb{Z}, \mathbb{Z}) = \{ 0 \}$, denn $0 = \varphi(0) =
  \varphi(1 + 1) = \varphi(1)+\varphi(1) \Rightarrow \varphi(1) = 0$
\end{nnBsp}

\begin{Bem}[Ähnlichkeiten von Moduln mit Vektorräumen]
  Die $R$-Moduln bilden eine \emp{abelsche Kategorie}\index{Kategorie!abelsche} \emp{$R$-Mod}\index{Kategorie!R-Mod}.
  \begin{enumerate}
    \item Eine Untergruppe $N$ eines $R$-Moduls $M$ heißt $R$-Untermodul von
          $M$, falls $R \cdot N \subseteq N$.
    \item Kern und Bild $R$-linearer Abbildungen sind $R$-Moduln.
    \item Zu jedem Untermodul $N \subseteq M$ gibt es einen Faktormodul $M/N$.
    \item Homomorphiesatz:\\
      Für einen surjektiven Homomorphismus $\varphi: M \rightarrow N$ gilt:
      $M/\Kern(\varphi) \cong N$.
    \item \emph{Direktes Produkt}: Sei ${\{M_{i}\}}_{i \in I}$ eine beliebige
          Menge von Moduln. Dann ist ihr direktes Produkt
     	  $\Pi_i M_i = \bigtimes_i M_i$ gegeben durch die Menge aller Tupel ${(m_i)}_{i
     	  \in I}$ mit $m_i \in M_i$ und die $R$-Aktion ${r(m_i)}_{i \in I} = {(rm_i)}_{i \in I}$.

	  \emph{Direkte Summe}: Das gleiche wie beim direkten Produkt, jedoch dürfen in den 
	  Tupeln nur endlich viele $m_i \neq 0$ sein.
  \end{enumerate}
\end{Bem}

\begin{Bew}
  \begin{enumerate}
    \stepcounter{enumi}
    \item $\Kern(\varphi)$: Sei $\varphi: M \rightarrow N$ lineare Abbildung. $m \in \Kern(\varphi)$, $r \in R$:\\
          $\varphi(rm) = r\varphi(m) = 0 \Rightarrow R \cdot \Kern(\varphi) \subseteq \Kern(\varphi)$; Untergruppe klar

	  $\Bild(\varphi)$: $n \in \Bild(\varphi) $, d. h. $\exists\, m: n = \varphi
   	  (m), m \in M \Rightarrow r \in R:
	  rn = r \varphi(m) = \varphi(rm) \in \Bild(\varphi)  \Rightarrow R
	  \cdot \Bild(\varphi) \subseteq \Bild(\varphi)$
    \item $M$ abelsch $\Rightarrow$ jedes $N$ Normalteiler $\Rightarrow M/N$ ist
          abelsche Gruppe.

    	  Wir definieren $R$-Aktion auf $M/N$ durch $r(m + N) = rm + N$. Das ist 
    	  wohldefiniert, denn\\
	  $r((m+n)+N)=r(m+n) + N= rm + \underbrace{rn}_{\in N} + N = rm + N$

	  $r((m+N) + (m' + N ) ) = r((m+m')+N) = r(m+m') + N = rm + N + rm' + N =
	  r(m+N) + r(m'+N)$\\
	  Die restlichen drei Eigenschaften gehen ähnlich.
	  
    \item
	$$\begin{xy}
              \xymatrix{
                M \ar[rr]^{\varphi} \ar[rd] &     &  N \\
                                            &  M/\Kern(\varphi) \ar@{-->}[ur]_{\exists!\, \tilde{\varphi}}  & }
          \end{xy}$$
	  Wohldefiniertheit von $\tilde{\varphi}$:\\
	  Sei $k \in \Kern(\varphi): \varphi(m+k) = \varphi(m)$

	  surjektiv: $\forall n \in N: n = \varphi(m) = \tilde{\varphi}(m + \Kern(\varphi))$

	  injektiv: $m, m' \in M$ mit $\varphi(m) = \varphi(m') = n \in N \Leftrightarrow 
	  \varphi(m-m') = 0 \Rightarrow m + \Kern(\varphi)(m) = \Kern(\varphi)(m')$

	  $\tilde{\varphi}$ ist $R$-linear: Klar, wegen $\varphi$ $R$-linear.
  \end{enumerate}
\end{Bew}

\begin{Bem}
  \begin{enumerate}
    \item Zu jeder Teilmenge $X \subseteq M$ eines $R$-Moduls $M$ gibt es den von
          $X$ erzeugten Untermodul $$\langle X \rangle = \displaystyle 
          \bigcap_{\substack{M' \subseteq M\; \text{ Untermodul} \\ X \subseteq M'}} M' = \left\{
          \sum_{i=1}^n a_i x_i: n \in \mathbb{N}, a_i \in R, x_i \in X \right\}$$
    \item $B \subset M$ heißt \emp{linear unabhängig}\index{linear unabhängig},
          wenn aus $\displaystyle \sum_{i=1}^n \alpha_i b_i = 0$ mit $n \in
          \mathbb{N}, b_i \in B, \alpha_i \in R$ folgt $\alpha_i = 0$ für alle
          $i$.
    \item Ein linear unabhängiges Erzeugendensystem heißt
          \emp{Basis}\index{Basis}.
    \item Nicht jedes $R$-Modul besitzt eine Basis.

          Beispiel: $\mathbb{Z}/2\mathbb{Z}$ als $\mathbb{Z}$-Modul: $\{\bar{1}\}$
          ist nicht linear unabhängig, da $\underbrace{42}_{\not= 0 \text{ in } \mathbb{Z}} \cdot 1 = 0$
    \item Ein $R$-Modul heißt \emp{frei}\index{R-Modul!freier}, wenn er eine
          Basis besitzt.
    \item Ein freier $R$-Modul $M$ hat die Universelle Abbildungseigenschaft eines Vektorraums. Ist $B$ eine
          Basis von $M$, $f: B \to M'$ eine Abbildung in einen $R$-Modul M', so
          gibt es genau eine $R$-lineare Abbildung $\varphi: M \to M'$ mit
          $\varphi|_B = f$.
    \item Sei $M$ freier Modul. Dann ist $M^*$ wieder frei und hat dieselbe
          Dimension wie $M$.
  \end{enumerate}
\end{Bem}

\begin{Bew}
  \begin{enumerate}
    \item[(f)] Sei $\{y_i\}_{i \in I}$ Familie von Elementen von $M'$.\\
      Sei $x \in M$. Durch $x=\sum_{i}a_ix_i$ ist $\{a_i\}_{i  \in I}$
      eindeutig bestimmt.\\
      Wir setzen: $\varphi(x):=\sum_i a_iy_i=\sum_ia_i\varphi(x_i)$

      \textbf{Beh. 1:} Falls $\{y_i\}_{i\in I}\;(y_i \neq y_j \text{ für } i\neq
      j)$ Basis von $M'$ ist, dann ist $\varphi$ ein Isomorphismus.\\
      \textbf{Bew. 1:} Wir können den Beweis des Satzes rückwärts anwenden\\
      $\Rightarrow$ $\exists\, \psi: M' \rightarrow M \text{ mit } \psi(y_i)=x_i \forall i \in I$\\
      $\Rightarrow$ $\varphi \circ \psi = id_N, \psi \circ \varphi = id_M$

      \textbf{Beh. 2:} Zwei freie Moduln mit gleicher Basis sind isomorph.\\
      \textbf{Bew. 2:} klar
  \end{enumerate}
\end{Bew}

\begin{PropDef}
  Sei $0 \to M' \overset{\alpha}{\to} M \overset{\beta}{\to} M'' \to 0$ kurze exakte Sequenz von $R$-Moduln (d.h.
  $M' \subseteq M$ Untermodul, $M'' = M/M'$). Dann gilt für jeden $R$-Modul $N$:
  \begin{enumerate}
    \item $0 \to \Hom_R(N,M') \overset{\alpha_*}{\to} \Hom_R(N,M) \overset{\beta_*}{\to}
          \Hom_R(N,M'')$ ist exakt.
    \item $0 \to \Hom_R(M',N) \overset{\beta^*}{\to} \Hom_R(M,N) \overset{\alpha^*}{\to}
          \Hom_R(M'',N)$ ist exakt.
    \item Im Allgemeinen sind $\beta_*$ bzw. $\alpha^*$ nicht surjektiv.
    \item Ein Modul $N$ heißt \emp{projektiv}\index{R-Modul!projektiver} (bzw.
          \emp{injektiv}\index{R-Modul!injektiver}), wenn $\beta_*$ (bzw.
          $\alpha^*$) surjektiv ist.
    \item Freie Moduln sind projektiv.
    \item Jeder $R$-Modul M ist Faktormodul eines projektiven $R$-Moduls.
    \item Jeder $R$-Modul M ist Untermodul eines injektiven $R$-Moduls.
  \end{enumerate}
\end{PropDef}

\begin{Bew}
  \begin{enumerate}
  \item $$
    \begin{xy}
      \xymatrix{
        &                    &  N \ar[ld]_\varphi \ar[d]^\psi \ar[dr] & & \\
	0 \ar[r] & M' \ar[r]^{\alpha} & M \ar[r]^{\beta} & M'' \ar[r] & 0
      }
    \end{xy}
    $$

    $\alpha_*$ ist injektiv: Sei $\varphi \in \Hom_R(N,M')$, ist
    $\alpha_*(\varphi) = \alpha \circ \varphi = 0 \overset{\alpha \text{ inj.}}{\Rightarrow} \varphi = 0$.

    $\Bild(\alpha_*) \subseteq \Kern(\beta_*)$:
    $\beta_*(\alpha_*(\varphi)) = \underset{=0}{\underbrace{\beta \circ \alpha}} \circ \varphi = 0$

    $\Kern(\beta_*) \subseteq \Bild(\alpha_*)$:\\
    Sei $\beta \circ \psi = 0$ ($\psi \in \Kern(\beta_*)$). Für jedes $x \in N$ ist $\psi(x) \in
    \Kern(\beta) = \Bild(\alpha) \Rightarrow$ zu $x \in N \;
    \exists\, y \in M' \text{ mit } \psi(x) = \alpha(y)$; $y$ ist
    eindeutig, da $\alpha$ injektiv.
    
    Definiere $\varphi': N \to M'$ durch $x \mapsto y$.\\
    Zu zeigen: $\varphi'$ ist $R$-linear\\
    Seien $x,x' \in N \Rightarrow \varphi'(x+x')=z$ mit $\alpha(z) =
    \varphi(x+x') = \varphi(x) + \varphi(x') = \alpha(y) + \alpha(y') =
    \alpha(y +y')$ mit $\varphi'(x) = y, \; \varphi'(x') = y'
    \overset{\alpha \text{ inj.}}{\Rightarrow} z = y + y'$

    Genauso: $\varphi'(a \cdot x) = a \cdot \varphi'(x)$
  \item \[
    \begin{xy}
      \xymatrix{
        0 \ar[r] & M' \ar[r] & M \ar[rr]^{\beta} \ar[rd]_{\beta^*(\varphi)} &  &  M'' \ar[dl]^{\varphi} \ar[r] & 0 \\
        & & & N & }
    \end{xy}
    \]
    $\beta^*$ injektiv, denn für $\varphi \in \Hom(M'', N)$ ist
    $\beta^*(\varphi)=\varphi\circ \beta$\\
    Sei $\beta^*(\varphi)= 0 \Rightarrow \varphi \circ \beta = 0 \overset{\beta
      \text{ surj.}}{\Rightarrow}\varphi=0$.

    $\Bild(\beta^*) \subseteq \Kern(\alpha^*)$: $(\alpha^* \circ 
    \beta^*)(\varphi)= \alpha^*(\varphi\circ \beta)=\varphi \circ
    \underbrace{\beta \circ \alpha}_{=0}=0$

    $\Kern(\alpha^*)\subseteq\Bild(\beta^*)$: Sei $\psi \in \Kern(\alpha^*)$.
    Aber $\psi \in \Hom_R(M, N)$ mit $\psi \circ \alpha=0$\\
    Weil $\psi$ auf $\Bild(\alpha)$ verschwindet, kommutiert
    \[
    \begin{xy}
      \xymatrix{
        & M'' &\\
          M \ar[rd]_{\psi} \ar[ur]^{\beta} \ar[rr] &     &  M/\Bild(\alpha)
        \ar[dl]^\sigma \ar[ul]_{\cong}\\
        &  N  & }
    \end{xy}
    \]
    $\Rightarrow \beta^*(\sigma)= \psi \Longrightarrow$ Beh.
  \item Im Allgemeinen sind $\beta_*$ und $\alpha^*$ nicht surjektiv\\
    z.B.: \begin{enumerate}
    \item[1.] $0\rightarrow \mathbb Z \stackrel{\cdot2}{\stackrel{\alpha}\rightarrow} 
      \mathbb Z \stackrel\beta\rightarrow \mathbb Z / 2\mathbb Z\rightarrow 0$ mit $N:= \mathbb Z / 2\mathbb Z$\\
      Es gilt: Hom$(N, \mathbb Z)=\{0\}$\\
      $\Hom(N, \mathbb Z/2\mathbb Z)=\{0, id\}$ $\Rightarrow$ $\beta_*$ nicht surjektiv $\Rightarrow$ $N$ nicht projektiv!
    \item[2.] $0\rightarrow \mathbb Z \stackrel{\cdot4}{\stackrel{\alpha}\rightarrow} 
      \mathbb Z \stackrel\beta\rightarrow \mathbb Z / 4\mathbb Z\rightarrow 0$ mit $N:= 2\cdot \mathbb Z / 4\mathbb Z$\\
      $\Hom(\mathbb Z, N)= \{0, \psi\}$, wobei $\psi(1)=2$.\\
      Dann: $\alpha^*(\psi)=\psi\circ \alpha = 0$ $\Rightarrow$ $\alpha^*$ nicht surjektiv $\Rightarrow$ $N$ nicht injektiv!
    \end{enumerate}
    \stepcounter{enumi}
  \item Sei $N$ frei mit Basis $\{e_i,i \in I\}$.
    Sei $\beta: M \to M''$ surjektive $R$-lineare Abbildung und
    $\varphi: N \to M''$ $R$-linear. Für jedes $i \in I$ sei $x_i \in M$
    mit $\beta(x_i) = \varphi(e_i)$ (so ein $x_i$ gibt es, da $\beta$
    surjektiv). Dann gibt es genau eine $R$-lineare Abbildung $\psi: N
    \to M$ mit $\psi(e_i) = x_i$. Damit $\beta(\psi(e_i)) = \beta(x_i) =
    \varphi(e_i)$ für alle $i \in I \Rightarrow \beta \circ \psi =
    \varphi$
  \item \label{1.5fBew}
    Sei $M$ ein $R$-Modul. Sei $X$ ein Erzeugendensystem von $M$ als
    $R$-Modul (notfalls $X = M$). Sei $F$ der freie $R$-Modul mit Basis
    $X$, $\varphi: F \to M$ die $R$-lineare Abbildung, die durch $x
    \mapsto x$ für alle $x \in X$ bestimmt ist. $\varphi$ ist surjektiv,
    da $X \subseteq \Bild(\varphi)$ und $\langle X \rangle = M$.
    Nach Homomorphiesatz ist $M \cong F/\Kern(\varphi)$.
  \end{enumerate}
\end{Bew}

\begin{Prop}
\label{1.6}
  Ein $R$-Modul $N$ ist genau dann projektiv, wenn es einen $R$-Modul $N'$ gibt,
  so dass $F \defeqr N \oplus N'$ freier Modul ist.
\end{Prop}

\begin{Bew}
  ,,$\Rightarrow$``:\\
  Sei $F$ freier $R$-Modul und $\beta: F \to N$ surjektiv (wie in \myref{Beweis von
  1.5}{1.5fBew}). Dann gibt es $\tilde{\varphi}: N \to F$ mit $\beta \circ
  \tilde{\varphi} = id_N$ (weil $N$ projektiv ist).\\
  \textbf{Behauptung:}
  \begin{enumerate}
    \item[1.)] $F = \Kern(\beta) \oplus \Bild(\tilde{\varphi}) \cong N' \oplus N$
    \item[2.)] $\tilde{\varphi}$ injektiv
  \end{enumerate}
  \textbf{Beweis:}
  \begin{enumerate}
    \item[1.)] $\Kern(\beta) \cap \Bild(\tilde{\varphi}) = (0)$, denn:
               $\beta(\tilde{\varphi}(x)) = 0 \Rightarrow x = 0 \Rightarrow
               \tilde{\varphi}(x) = 0$.
  
               Sei $x \in F,\; y \defeqr
               \tilde{\varphi}(\beta(x)) \in \Bild(\tilde{\varphi})$.
               Für $z = x - y$ ist $\beta(z) = \beta(x) -
               \underbrace{\beta(\tilde{\varphi}}_{id}(\beta(x)))= 0 \Rightarrow x = \underbrace{z}_{\in
               \Kern(\beta)} + \underbrace{y}_{\in \Bild(\tilde{\varphi})}$
    \item[2.)] $\tilde{\varphi}(x) = 0 \Rightarrow \underbrace{\beta(\tilde{\varphi}(x))}_{= x} = 0$
  \end{enumerate}
  ,,$\Leftarrow$``:\\
  Sei $F = N \oplus N'$ frei, $\beta: M \to M''$ surjektiv, $\varphi: N \to M''$
  R-linear.

  Gesucht: $\psi: N \to M$ mit $\beta \circ \psi = \varphi$.

  Definiere $\tilde{\varphi}: F \to M''$ durch $\tilde{\varphi}(x + y) =
  \varphi(x)$ wobei jedes $z \in F$ eindeutig als $z = x + y$ mit $x \in N,\; y
  \in N'$ geschrieben werden kann.

  $F$ ist frei also projektiv $\Rightarrow \exists\, \tilde{\psi}: F \to M$ mit
  $\beta \circ \tilde{\psi} = \tilde{\varphi}$. Sei $\psi \defeqr
  \tilde{\psi}|_N$. Dann ist $\beta \circ \psi = \beta \circ \tilde{\psi}|_N =
  \tilde{\varphi}|_N = \varphi$
\end{Bew}
