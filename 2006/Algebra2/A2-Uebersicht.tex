\documentclass[11pt]{article}
 
% Spracheinstellung
\usepackage[latin1]{inputenc}
\usepackage[ngerman]{babel}

% Type1 Font: Times
\usepackage[T1]{fontenc}
\usepackage{times}

% Textsatz ohne Einr�ckung aber mit Abstand
\setlength{\parskip}{0pt}
\setlength{\parindent}{0pt}

% Seitengr��e und R�nder ohne Seitenzahlen
\usepackage[a4paper,tmargin=2.0cm,bmargin=2.0cm,lmargin=2cm,rmargin=2cm]{geometry}
\pagestyle{empty}

\usepackage{amsmath,amsfonts,stmaryrd}


% URL Link
\usepackage[colorlinks=true, linkcolor=blue, urlcolor=black]{hyperref}

% Abstand �berschriften zum Text
\usepackage{titlesec}
\titlespacing{\section}{0pt}{*1}{*.5}
\titlespacing{\subsection}{0pt}{*1}{*0}
\titleformat{\section}[hang]{\bfseries\large}{\thesection}{1em}{}
\titleformat{\subsection}[hang]{\bfseries}{\thesubsection}{1em}{}

% Metadaten
\title{Algebra 2 �bersicht}
\hypersetup{pdftitle={Algebra 2 �bersicht}}

\begin{document}

\centerline{\Large Algebra 2 �bersicht}

\section{Multilineare Algebra}

\subsection{Moduln}

$R$-Modul Grunddefinitionen und Beispiele

$R$-Modul Grundeigenschaften

Def: freier Modul

Projektive Moduln

Beweis: Sequenz projektiver Modul

Beweis: Sequenz injektiver Modul

Beispiele f�r nicht projektive / injektive Modul

Beweis: Freie Moduln sind projektiv. Jeder $R$-Modul ist Faktormodul ...

Prop+Beweis: proj. Modul $\Leftrightarrow$ $F = N \oplus N'$ frei

\subsection{Tensorprodukt}

Tensorprodukt: Definition und Beispiele

Satz 1: Tensorprodukt

Tensorprodukt-�quivalenzen und Beweis

Tensorprodukt mit Idealen

\subsection{Flache Moduln}

Funktor $\otimes_R N$ mit Beispiel

Flacher Modul und 3 Bemerkungen

Projektive Moduln sind flach (drei-teiliger Beweis)

\subsection{Tensoralgebra}

Def: Algebra, 2 Funktoren, Beispiele

UAE von $R' \otimes_R R''$

Def: Tensoralgebra

\subsection{Symmetrische und �u�ere Algebra}

Def: symmetrische und alternierende Abbildungen

Satz 2: symmetrische und �u�ere Potenz

Satz: Struktur von $S^n(M)$ und $\Lambda^n(M)$

\subsection{Differentiale}

Def: Derivation mit Beispielen

Darstellbarkeit von $M \mapsto \mbox{Der}_R(A,M)$

2 Beispiele: $\Omega_{R[X_1,\ldots,X_n]/R}$ $\Omega_{\mathcal{C}^\infty(X)/R}$ 

$\Omega_{\cdot/R}$ ist rechtsexakter Funktor

\subsection{Der de\,Rham-Komplex}

Def+Satz: de\,Rham-Komplex

2 Beispiele zu $H_{dR}^i(A)$

\section{Noethersche Ringe und Moduln}

\subsection{Der Hilbertsche Basissatz}

Def+Bsp: noethersch

Satz+Bew: $0 \rightarrow M' \rightarrow M \rightarrow M'' \rightarrow 0$ noethersch �quivalenz

Satz: Endlich erzeugbare Moduln �ber noetherschen Ringen

Drei �quivalenzen zu noethersch

Satz: Hilbertsche Basissatz + zwei Folgerungen

\subsection{Ganze Ringerweiterungen}

Def: ganz, normiert, ganzer Abschluss, ganz abgeschlossen, normal; Satz zu normal

�quivalenzen zu ganz

\subsection{Der Hilbertsche Nullstellensatz}

Satz 5: Hilbertsche Nullstellensatz

Def+Bsp: Verschwindungsideal, Nullstellenmenge + 2 Beispiele

Satz: Schwacher Nullstellensatz

Satz: Starker Nullstellensatz

\subsection{Graduierte Ringe und Moduln}

Def: graduierter Ring, homogen; 1 Bemerkung

Def: homogenes Ideal; 2 Bemerkungen

3 Beispiele zu homogenenn Idealen

Drei �quivalenzen f�r noethersche graduierte Ringe

Def: graduierter $S$-Modul, graderhaltend, $\mbox{Grad}(\varphi)$, Twist

Satz: $\dim S_d^{(n)} = \cdots$ in $S = K[X_1,\ldots,X_n]$

Satz: Hilbert-Polynom

Def: Hilbert-Reihe mit 3 Beispielen

Satz 6': Hilbert-Reihe

\subsection{Invarianten endlicher Gruppen}

Def: Invariantenring, linear; 2 Beispiele

Satz: Endliche Erzeugbarkeit des Invariantenrings

Beispiel eines Invariantenrings

\subsection{Nakayama, Krull und Artin-Rees}

Def+Bem+Bsp: Jacobson Radikal

Satz: Lemma von Nakayama + Folgerungen und Beispiel

Proposition: Artin-Rees

Satz 9: Durchschnittssatz von Krull

Beispiele zu Artin-Rees

\subsection{Krull-Dimension}

Def+Bsp: Krull-Dimension

Bem: Krull-Dimension von nullteilerfreien Ringen

Satz 10: Primidealketten in Ringerweiterungen

Folgerung �ber Maximalit�t von Primidealen in Ringerweiterungen

Satz 11: Noether-Normalisierung

\subsection{Das Spektrum eines Rings}

Def+Bsp+Bem: Spektrum eines Rings

3 Bem zu Spektrum, Zariski-Topologie

Def: irreduzibel, Prop dazu mit Primideal

Folg: mit hausdorffsch, Def Verschwindungsideal + Folgerung

Def+Bem: Irreduzible Komponente

Prop: Spec als Funktor: Auswirkungen eines Ringhomomorphismus

\subsection{Diskrete Bewertungsringe}

Def: diskrete Bewertung + Beispiele

Def: Absolutbetrag und Geometrie: Kreis und Dreieck

Def+Bem: $\mathbb{Q}_p$ und $\mathcal{O}_v$ lokaler Ring

Def+Prop: Diskreter Bewertungsring

Satz 12: Diskrete Bewertungsringe + 1 Gegenbeispiel

\subsection{Dedekindringe}

Def: Dedekindring + 4 Beispiele

Def: gebrochenes Ideal, invertierbares Ideal, 5 Beispiele, 1 Bemerkung

Satz 13: Dedekindringe

Satz 14: Dedekindring und $L/K$ K�rpererweiterung

\subsection{Prim�rzerlegung}

Def+Bem+Bsp: Prim�rideal

Def: $\mathfrak{p}$-prim�r Prim�rzerlegung, reduzierte

Satz 15: Reduzierte Prim�rzerlegung

\end{document}