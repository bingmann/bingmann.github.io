\section{Dedekindringe}

\begin{Def}

Ein nullteilerfreier Ring heißt \emp{Dedekindring}\index{Dedekindring}, wenn er noethersch, normal und eindimensional ist.

\begin{nnBsp}
\begin{enumerate}
\item[1)] $\mathbb{Z}$, $k[X]$ ($k$ Körper)

\item[2)] diskrete Bewertungsringe

\item[3)] Hauptidealringe (nullteilerfrei)

\item[4)] der ganze Abschluss $\mathcal{O}_d$ von $\mathbb{Z}$ in $\mathbb{Q}(\sqrt{d})$ wobei $d \in \mathbb{Z}$ quadratfrei.

$\mathcal{O}_d = \begin{cases}
\mathbb{Z}[\sqrt{d}] & d \not\equiv 1 \mod 4\\
\mathbb{Z}[\frac{1+\sqrt{d}}{2}] & d \equiv 1 \mod 4
\end{cases}$

\end{enumerate}
\end{nnBsp}

\end{Def}

Beobachtung: Es gibt Dedekindringe, die nicht faktoriell sind: Beispiel: $\mathbb{Z}[\sqrt(-5)]$.\\
($2 \cdot 3 = (1 + \sqrt{-5}) (1 - \sqrt{-5})$.

\begin{DefBem}
Sei $R$ nullteilerfrei, $K = \textrm{Quot}(R)$
\begin{enumerate}
\item[a)] Ein $R$-Untermodul $I \neq (0)$ von $K$ heißt \emp{gebrochenes Ideal}\index{Ideal!gebrochenes} von $R$, wenn es ein $a \in R \setminus \{0\}$ gibt mit $a \cdot I \subseteq R$. (Beispiel: $n \cdot (\frac{1}{n})$ mit $R = \mathbb{Z}$)

\item[b)] Für gebrochene Ideale $I,J$ von $R$ sei $I \cdot J$ der von allen $a \cdot b$, $a \in I, b \in J$, erzeugte $R$-Untermodul von $K$.

\item[c)] Die gebrochenen Ideale von $R$ bilden mit der Multiplikation aus b) ein kommutatives Monoid mit neutralem Element $R$.

\item[d)] Die Einheiten in diesem Monoid heißen \emp{invertierbare} (gebrochene) Ideale.

d.h. $I$ invertiertbar $\Leftrightarrow$ $\exists I'$ mit $I \cdot I' = R$.

\end{enumerate}
\end{DefBem}

\begin{nnBsp}
\begin{enumerate}
\item[0)] Jedes Ideal in $R$.

\item[1)] Jeder endlich erzeugbare $R$-Untermodul von $K$ ist gebrochenes Ideal.

\textbf{denn:} Seien $x_1 = \frac{a_1}{b_1}, \ldots, x_n = \frac{a_n}{b_n}$ Erzeuger von $M$ ($a_i, b_i \in R$) $\Rightarrow$ für $b = b_1 \cdot \ldots \cdot b_n$ ist $b \cdot M \subseteq R$.

\item[2)] Ist $I$ gebrochenes Ideal, so ist $I^{-1} := \{ x \in K : x \cdot I \subseteq R \}$ ebenfalls gebrochenes Ideal: für jedes $a \in I$ ist $a \cdot I^{-1} \subseteq R$.

$I$ ist invertierbar $\Leftrightarrow$ $I \cdot I^{-1} = R$.

\item[3)] $R = k[X,Y]$, $I = (X,Y)$ $\Rightarrow$ $I^{-1} = R$.

\textbf{denn:} für $a = \frac{f}{g} \in I^{-1}$ muss gelten: $a \cdot X \in R$, $a \cdot Y \in R$.

\item[4)] Jedes Hauptideal $\neq (0)$ ist invertierbar: $(a) \cdot (\frac{1}{a} \cdot R) = R$.
\end{enumerate}
\end{nnBsp}

\begin{Bem}\label{2.41}
Jedes invertierbare Ideal in einem Integritätsbereich ist endlich erzeugbar (als $R$-Modul).

\begin{Bew}
Sei $I$ invertierbar, also $I \cdot I^{-1} = R$, dann gibt es $a_i \in I, b_i \in I^{-1}$ mit $1 = \sum_{i=1}^{n} a_i b_i$

\textbf{Beh:} $a_1, \ldots a_n$ erzeugen $I$.

\textbf{denn:} Sei $a \in I$ $\Rightarrow$ $a = a \cdot 1 = a \cdot \sum_{i=1}^{n} a_i b_i = \sum_{i=1}^{n} a_i \underbrace{(a b_i)}_{\in R}$

\end{Bew}
\end{Bem}

\begin{Satz}[Dedekindringe]\label{Satz13}
Für einen nullteilerfreien Ring $R$ sind äquivalent:

\begin{enumerate}
\item[(i)] $R$ ist Dedekindring oder Körper.

\item[(ii)] $R$ ist noethersch und $R_\mathfrak{p}$ ist diskreter Bewertungsring für jedes Primideal $\mathfrak{p} \neq (0)$ in $R$.

\item[(iii)] Jedes Ideal $I \neq (0)$ in $R$ ist invertierbar.

\item[(iv)] Die gebrochenen Ideal in $R$ bilden eine Gruppe.

\item[(v)] Jedes echte Ideal in $R$ ist Produkt von endlich vielen Primidealen.

\item[(vi)] Jedes echte Ideal besitzt eine eindeutige Darstellung als Produkt von endlich vielen Primidealen.
\end{enumerate}

\end{Satz}

\begin{Bew}

\textbf{Beweisplan:}
\[
\begin{xy}
\xymatrix{
                 & (ii) \ar@{=>}[dr] &                              & (v) \ar@{<=>}[dd] \\
(i) \ar@{=>}[ur] &                   & (iii) \ar@{=>}[dl] \ar@{<=>}[ur] \\
                 & (iv) \ar@{=>}[ul] &                              & (vi) 
}
\end{xy}
\]

\begin{description}
\item[(i) $\Rightarrow$ (ii)]:

Sei $\mathfrak{p} \neq (0)$ Primideal im Dedekindring $R$. $\Rightarrow$ $R_\mathfrak{p}$ noethersch, $\dim R_\mathfrak{p} = \textrm{lat}(\mathfrak{p}) = 1$, da $\dim R = 1$.

$R_\mathfrak{p}$ normal: Sei $a \in K = \textrm{Quot}(R) = \textrm{Quot}(R_\mathfrak{p})$ ganz über $R_\mathfrak{p}$.

Dann gibt es eine Gleichung: $a^n + \sum_{i=0}^{n-1} \frac{b_i}{s_i} a^i = 0$ mit $b_i \in R, s_i \in R \setminus \mathfrak{p}$

$\Rightarrow$ $(s \cdot a)^n + \sum_{i=0}^{n-1} \widetilde{b_i} (s a)^i = 0$ mit $\widetilde{b_i} \in R$, $s := \prod_{i=0}^{n-1} s_i$

$\underset{R \text{ normal}}{\Longrightarrow}$ $s \cdot a \in R$ $\Rightarrow$ $a \underset{s \notin \mathfrak{p}}{=} \frac{s \cdot a}{s} \in R_\mathfrak{p}$

\item[(iii) $\Rightarrow$ (iv)]:

Sei $(0) \neq I \subset K$ gebrochenes Ideal, $a \in R \setminus \{0\}$ mit $a \cdot I \subseteq R$. $\underset{\text{(iii)}}{\Rightarrow}$ $a \cdot I$ invertierbar. $\Rightarrow$ $R = (a \cdot I) \cdot I' = I \cdot (a \cdot I')$ $\Rightarrow$ $I$ ist invertierbar.

\item[(ii) $\Rightarrow$ (iii)]:

Sei $I \neq (0)$ Ideal in $R$. $K = \textrm{Quot}(R)$. $I^{-1} := \{ x \in K : x \cdot I \subseteq R \}$
	
Zu zeigen: $I \cdot I^{-1} = R$.

Annahme: $I \cdot I^{-1} \subsetneqq R$:\\
Dann gibt es ein maximales Ideal $\mathfrak{m}$ von $R$ mit $I \cdot I^{-1} \subseteq \mathfrak{m}$.\\
$\Rightarrow$ $R_\mathfrak{m}$ ist diskreter Bewertungsring.\\
$\Rightarrow$ $I \cdot R_\mathfrak{m}$ ist Hauptideal, d.h. $I \cdot R_\mathfrak{m} = \frac{a}{s} \cdot R_\mathfrak{m}$ für ein $a \in I, s \in R \setminus \mathfrak{m}$

Seien $b_1, \ldots, b_n \in I$ Erzeuger ($R$ ist noethersch). $\Rightarrow$ $\frac{b_i}{1} = \frac{a}{s} \cdot \frac{r_i}{s_i}$ für gewisse $r_i \in R, s_i \in R \setminus \mathfrak{m}$

Sei $t = s \cdot \prod_{i=1}^{n} s_i$. Es gilt: $t \in R \setminus \mathfrak{m}$.

Für jedes $i = 1, \ldots n$ ist $\frac{t}{a} \cdot b_i = r_i \cdot s_i \cdot \ldots \cdot \widehat{s_i} \cdot \ldots \cdot s_n \in R$.

$\Rightarrow$ $\frac{t}{a} \in I^{-1}$ $\Rightarrow$ $t = a \cdot \frac{t}{a} \in I \cdot I^{-1} \subseteq \mathfrak{m}$. Widerspruch.

\item[(iv) $\Rightarrow$ (i)]:

\underline{$R$ noethersch}: Nach \myref{Bemerkung}{2.41} ist jedes invertierbare Ideal endlich erzeugbar.

\underline{$R$ normal}: Sei $x \in K$ ganz über $R$. $\Rightarrow$ $R[x]$ ist endlich erzeugbarer $R$-Modul, also gebrochenes Ideal (Beispiel 1). $\underset{\text{(iv)}}{\Rightarrow}$ $R[x]$ ist invertierbar. 

Da $R[x]$ Ring ist, gilt $R[x] \cdot R[x] = R[x]$. $\underset{R[x]\text{ invertierbar}}{\Longrightarrow}$ $R[x] = R$ (neutrale Element).

$\Rightarrow$ $x \in R$.

\underline{$\dim R \leq 1$}: Sei $\mathfrak{p} \neq (0)$ Primideal in $R$, $\mathfrak{m} \subseteq R$ maximales Ideal mit $\mathfrak{p} \subseteq \mathfrak{m}$.

$\Rightarrow$ $\mathfrak{m}^{-1} \cdot \mathfrak{p} \subseteq \mathfrak{m}^{-1} \mathfrak{m} \underset{\text{(iv)}}{=} R$ und $\mathfrak{m} \cdot (\mathfrak{m}^{-1} \mathfrak{p}) = \mathfrak{p}$.

$\underset{\mathfrak{p}\text{ Primideal}}{\Longrightarrow}$ $\mathfrak{m} = \mathfrak{p}$ oder $\mathfrak{m}^{-1} \mathfrak{p} \subseteq \mathfrak{p}$.

Falls $\mathfrak{m}^{-1} \mathfrak{p} \subseteq \mathfrak{p}$ $\underset{\cdot \mathfrak{p}^{-1}}{\Rightarrow}$ $\mathfrak{m}^{-1} \subseteq R$. Widerspruch (da sonst $\mathfrak{m}^{-1} \cdot \mathfrak{m} \subseteq \mathfrak{m}$)

\item[(iii) $\Rightarrow$ (v)]:

Sei $I \neq (0)$, $I \neq R$ Ideal in $R$.

Setze $I_0 := I$.

Definiere induktiv: $I_n$ für $n \geq 1$:

Ist $I_{n-1} \neq R$, so sei $\mathfrak{m}_{n-1}$ maximales Ideal mit $I_{n-1} \subseteq \mathfrak{m}_{n-1}$ und $I_n := I_{n-1} \mathfrak{m}_{n-1}^{-1} \subseteq R$.

Es ist $I_{n-1} \subseteq I_n$

Wäre $I_n = I_{n-1}$, so wäre $\mathfrak{m}_{n-1}^{-1} = R$. Widerspruch zu $\mathfrak{m}_{n-1}^{-1} \cdot \mathfrak{m}_{n-1} = R$.

Da nach \ref{2.41} $R$ noethersch ist, wird die Kette $I_0 \subsetneqq I_1 \subsetneqq I_2 \subsetneqq \cdots$ stationär.

$\Rightarrow$ $\exists n$ mit $R = I_n = I_{n-1} \mathfrak{m}_{n-1}^{-1} = I_{n-2} \mathfrak{m}_{n-2}^{-1} \mathfrak{m}_{n-1}^{-1} = \cdots = I_0 \cdot \prod_{i=0}^{n-1} \mathfrak{m}_{i}^{-1}$

$\Rightarrow$ $I = I_0 = \prod_{i=0}^{n-1} \mathfrak{m}_i$

\item[(v) $\Rightarrow$ (vi)]:

Sei $\mathfrak{p}_1 \cdots \mathfrak{p}_n = \mathfrak{q}_1 \cdots \mathfrak{q}_m$ mit Primidealen $\mathfrak{p}_i$, $\mathfrak{q}_i$. Zu zeigen: $n=m$ und $\mathfrak{p}_i = \mathfrak{q}_{\sigma(i)}$ für eine Permutation $\sigma \in S_n$:

Induktion über $n$:

$n=1$: $\mathfrak{p} = \mathfrak{p}_1 = \mathfrak{q}_1 \cdots \mathfrak{q}_m$ $\underset{\mathfrak{p}\text{ prim}}{\Rightarrow}$ $\exists\, i_0$ mit $\mathfrak{q}_{i_0} \subseteq \mathfrak{p}$. Umgekehrt ist $\mathfrak{p} \subseteq \mathfrak{q}_i$ für jedes $i$. $\Rightarrow$ $\mathfrak{p} = \mathfrak{q}_{i_0}$

$n>1$: Ohne Einschränkung $\mathfrak{p}_1$ minimal bzgl. $\subseteq$ in $\{ \mathfrak{p}_1, \ldots \mathfrak{p}_n \}$.

Aus $\prod \mathfrak{q}_i \subseteq \prod \mathfrak{p}_j \subseteq \mathfrak{q}_{i_1}$ $\Rightarrow$ $\exists j_0$ mit $\mathfrak{p}_{j_0} \subseteq \mathfrak{q}_{i_0} \subseteq \mathfrak{p}_1$ $\underset{\mathfrak{p}_1\text{ minimal}}{\Rightarrow}$ $\mathfrak{p}_1 = \mathfrak{q}_{i_0}$ $\underset{\text{(iii)}}{\Rightarrow}$ $\mathfrak{p}_2 \cdots \mathfrak{p}_n = \mathfrak{q}_1 \ldots \widehat{\mathfrak{q}_{i_0}} \ldots \mathfrak{q}_m$ $\Rightarrow$ Behauptung aus Induktionsvoraussetzung.

\item[(v) $\Rightarrow$ (iii)]:

Sei $I \neq (0)$, $I = \mathfrak{p}_1 \cdots \mathfrak{p}_r$ mit Primidealen $\mathfrak{p}_i$. Ist jedes $\mathfrak{p}_i$ invertierbar, so ist $I^{-1} = \mathfrak{p}_1^{-1} \ldots \mathfrak{p}_r^{-1}$ und $I \cdot I^{-1} = R$. Also ohne Einschränkung $I = \mathfrak{p}$ Primideal.

Sei $a \in \mathfrak{p} - \{0\}$, $(a) = \mathfrak{q}_1 \ldots \mathfrak{q}_n$ mit Primidealen $\mathfrak{q}_i$ $\Rightarrow$ $\mathfrak{q}_i \subseteq \mathfrak{p}$ für ein $i$.

$\mathfrak{q}_i$ ist invertierbar: $\mathfrak{q}_i^{-1} = \frac{1}{a} \cdot R \cdot \mathfrak{q}_1 \cdots \widehat{\mathfrak{q}_i} \cdots \mathfrak{q}_n$

Es genügt also zu zeigen: $\mathfrak{q}_i = \mathfrak{p}$

\textbf{Beh. 1:} Jedes invertierbare Primideal $\mathfrak{q}$ in $R$ ist maximal.

\textbf{Bew. 1:}
Ist $\mathfrak{q}$ nicht maximal, so sei $x \in R \setminus \mathfrak{q}$ mit $\mathfrak{q} + (x) \neq R$.

\textbf{Beh. 2:} Dann ist $(\mathfrak{q} + (x))^2 = \mathfrak{q} + (x^2)$

Dann ist $\mathfrak{q} \subseteq \mathfrak{q} + (x^2) \underset{\text{Beh. 2}}{=} (\mathfrak{q} + (x))^2 \subseteq \mathfrak{q}^2 + (x)$ $(\ast)$

Weiter ist $\mathfrak{q} \subseteq \mathfrak{q}^2 + \mathfrak{q} \cdot (x)$

\textbf{denn:} Sei $b \in \mathfrak{q}$, schreibe nach $(\ast)$ $b = c + r x$ mit $c = \mathfrak{q}^2, r \in R$, dabei ist $r \in \mathfrak{q}$, da $r \cdot x \in \mathfrak{q}$ und $x \notin \mathfrak{q}$.

$\Rightarrow$ $\mathfrak{q} = \mathfrak{q}^2 + \mathfrak{q} \cdot (x)$ (,,$\supseteq$`` ist trivial)

$\Rightarrow$ $\mathfrak{q} = \mathfrak{q} (\mathfrak{q} + (x)) \underset{\mathfrak{q}\text{ invertierbar}}\Rightarrow R = \mathfrak{q} + (x)$ Widerspruch.

\textbf{Bew. 2:} ,,$\subseteq$`` \checkmark, ,,$\supseteq$``

Schreibe beide Seiten als Produkt von Primidealen.

$\mathfrak{q} + (x) = \mathfrak{p}_1 \cdots \mathfrak{q}_r$, $\mathfrak{q} + (x^2) = \mathfrak{q}_1 \cdots \mathfrak{q}_s$.

In $R / \mathfrak{q}$ ist dann: $(\bar{x}) = \bar{\mathfrak{p}}_1 \cdots \bar{\mathfrak{p}}_r$, $(\bar{x})^2 = \bar{\mathfrak{q}}_1 \cdots \bar{\mathfrak{q}}_s = \bar{\mathfrak{p}}_1^2 \cdots \bar{\mathfrak{q}}_r^2$

$(\bar{x}), (\bar{x}^2)$ invertierbar. $\Rightarrow$ $\bar{\mathfrak{p}_i}, \bar{\mathfrak{q}_j}$ invertierbar.

$\underset{\text{,,(iii) + (v) = (vi)``}}{\Rightarrow}$ $\bar{\mathfrak{q}}_i = \bar{\mathfrak{p}}_{\sigma(i)}^2$ $\Rightarrow$ ohne Einschränkung $\mathfrak{q}_i = \mathfrak{p}_i^2$.

\end{description}
\end{Bew}

\begin{Satz} 
Sie $R$ ein Dedekindring, $K = \Quot(R), \; L/K$ endliche separable
Körpererweiterung.
$S$ der ganze Abschluß von $R$ in $L$.\\
Dann ist $S$ ein Dedekindring.
\end{Satz}

\begin{Bew} 
\underline{$\dim S=1:$} Folgt aus \hyperref[Satz10]{Satz~\ref*{Satz10}\ref*{Satz10c}}

\underline{$S$ normal:}\\
Sei $x\in L$ ganz über $S$, also $x^n+\sum_{i=1}^{n-1}a_i x^i = 0$ mit $a_i \in S$.
Sei $S'$ der von $R$ und $a_1,\dots,a_{n-1}$ erzeugte Unterring von $S$.
$S'$ ist endlich erzeugbarer $R$-Modul, da die $a_i$ ganz über $R$ sind.
$S[X]$ ist endlich erzeugter $S'$-Modul und damit endlich erzeugbarer $R$-Modul $\Rightarrow x$ ist ganz über $R \Rightarrow x \in S$.

\underline{$S$ noethersch:}\\
\textbf{Beh. 1:} Es gibt ein primitives Element $\alpha$ von $L/K$ mit $\alpha \in S$.\\
\textbf{Bew. 1:} Sei $\tilde{\alpha} \in L$ primitives Element, also $1, \tilde{\alpha}, \tilde{\alpha}^2, \dots, \tilde{\alpha}^{n-1}$ ist $K$-Basis von $L \; (n \defeqr [L:K])$.
Sei $\tilde{\alpha} = \sum_{i=0}^{n-1} c_i \tilde{\alpha}^i$ für gewisse $c_i \in K, \; i = 0, \dots, n-1$.
Schreibe $c_i = \frac{a_i}{b_i}$ mit $a_i, b_i \in R, \; b \defeqr \prod_{i=0}^{n-1} b_i$.
Setze $\alpha \defeqr b \cdot \tilde{\alpha} \Rightarrow \alpha^n = b^n \cdot
\sum_{i=0}^{n-1} c_i \tilde{\alpha}^i = \sum_{i=0}^{n-1} \underset{\in
R}{\underbrace{c_i b^{n-i}}}\alpha^i \Rightarrow \alpha \in S$

$1, \alpha, \alpha^2, \dots, \alpha^{n-1}$ linear unabhängig:\\
Sei $\sum_{i=0}^{n-1} \lambda_i \alpha^i = 0 \Rightarrow \sum \lambda_i b^i \tilde{\alpha}^i = 0 \Rightarrow \lambda_i b^i = 0 \; \forall i$

Sei nun $\bar{K}$ ein algebraischer Abschluss von K.
Seien $\sigma_1, \dots, \sigma_n$ die verschiedenen Einbettungen von $L$ in
$\bar{K}$, also die Elemente von $\Hom(L,\bar{K})$.\\
$d \defeqr d(\alpha) \defeqr (\det(\sigma_i(\alpha^{j-1})_{i,j=1, \dots, n}))^2$
heißt die Diskriminante von $L/K$ (bzgl. $\alpha$).

\textbf{Beh. 2:}
\vspace{-1.5ex}
\begin{enumerate} 
  \item $d \not= 0$
  \item $S$ ist in dem von $\frac{1}{d}, \frac{\alpha}{d}, \dots,
  \frac{\alpha^{n-1}}{d}$ erzeugten $R$-Untermodul von $L$ enthalten.
\end{enumerate}
Dann ist $S$ als Untermodul eines endlich erzeugbaren $R$-Modul selbst endlich
erzeugbar und damit noethersch (weil $R$ noethersch ist).

\textbf{Bew. 2:}
\begin{enumerate}
\item $d = \det
  \begin{pmatrix}
    1 & 1 & \dots & 1 \\
    \sigma_1(\alpha) & \sigma_2(\alpha) & \dots & \sigma_n(\alpha) \\
    \sigma_1(\alpha)^2 & \sigma_2(\alpha)^2 & \dots & \sigma_n(\alpha)^2 \\
    \vdots & \vdots & \ddots & \vdots \\
    \sigma_1(\alpha)^{n-1} & \sigma_2(\alpha)^{n-1} & \dots & \sigma_n(\alpha)^{n-1}
  \end{pmatrix}
  \overset{\text{Vandermonde}}{=} \displaystyle\prod_{i \not= j}
  (\sigma_i(\alpha) - \sigma_j(\alpha)) \not= 0$

\item Für $x \in L$ sei $\text{Spur}(x) \defeqr \sum_{i=1}^n \sigma_i(x) \in \bar{K}$

  $\text{Spur}(x) \in K:$ Für $\sigma \in \Aut_K(\bar{K})$ ist $\sigma \circ \sigma_i \in \Hom_K(L,\bar{K})$

  $\sigma(\text{Spur}(x)) = \sum_{i=1}^n (\sigma \circ \sigma_i)(x) = \text{Spur}(x) \in \bar{K}^{\Aut_K(\bar{K})} = K$.

  Sei $x \in S, \; x = \sum_{j=1}^n c_j \alpha^j$ mit $c_j \in K$.
\end{enumerate}

\textbf{Beh. 3:} $c = \begin{pmatrix} c_1\\ \vdots\\ c_n \end{pmatrix}$ ist
Lösung eines LGS $A \cdot c = b$ mit $b \in R^n$ und $A \in R^{n \times n}$
mit $\det A = d$.\\
Nach der Cramerschen Regel ist dann $c_i = \frac{\det A_i}{\det A}$ wobei
$A_i$ aus $A$ dadurch entsteht, dass die $i$-te Zeile durch $b$ ersetzt wird.
$\Rightarrow$ $c_i \in \frac{1}{d}R$ $\Rightarrow$ $x$ liegt in 
dem von $\frac{1}{d}, \frac{\alpha}{d}, \dots, \frac{\alpha^{n-1}}{d}$ erzeugten $R$-Modul.

\textbf{Bew. 3:} Für $i=1, \dots, n$ ist $\text{Spur}(\alpha^{i-1} x) = \sum_{j=1}^n \text{Spur}((\alpha^{i-1}\alpha^{j-1})c_j) \in K \quad (*)$ ganz über $R$\\
$\Rightarrow$ $\text{Spur}(\alpha^{i-1}x) \in R \Rightarrow A \defeqr (\text{Spur}(\alpha^{i-1} \alpha^{j-1})_{i,j = 1, \dots, n}) \in R^{n \times n}$

$b \defeqr
\begin{pmatrix}
  \text{Spur}(x) \\
  \text{Spur}(\alpha x) \\
  \vdots \\
  \text{Spur}(\alpha^{n-1}x)
\end{pmatrix} \in R^n$ $(*)$ heißt $A \cdot c = b$.

Noch zu zeigen: $\det A = d$.\\
Nach Definition ist $d = (\det B)^2$ mit $B = (\sigma_i(\alpha^{j-1})_{i,j})$\\
$\Rightarrow$ $B^T \cdot B = (\beta_{ij})$ mit $\beta_{ij} = \sum_{k=1}^n \sigma(\alpha^{i-1}) \sigma_k(\alpha^{j-1}) = \text{Spur}(\alpha^{i-1} \alpha^{j-1})$\\
$\Rightarrow$ $B^T \cdot B = A$ $\Rightarrow$ $\det A = (\det B)^2 = d$
\end{Bew}

\begin{nnBsp}
$K=\mathbb{Q}$, $L=\mathbb{Q}(\sqrt{D})$, $D$ quadratfrei, $R=\mathbb{Z}$.

Was ist $d$? $\alpha = \sqrt{D}$, $\sigma_1=\textrm{id}$, $\sigma_2(a+b\sqrt{D})=a-b\sqrt{D}$

$B=\left(\begin{array}{cc}1&1\\\sqrt{D}&-\sqrt{D}\end{array}\right)$

$d=(\textrm{det}\ B)^2=(-2\sqrt{D})^2=4D$

\end{nnBsp}
